\documentclass[12pt, a4paper, twoside]{article}
\usepackage[left = 3cm, top = 3cm, right = 2cm, bottom = 2cm]{geometry}
\usepackage[brazilian]{babel}
\usepackage[utf8]{inputenc}
\usepackage{amsmath, amsfonts, amssymb}
\numberwithin{equation}{subsection} %subsection
\usepackage{fancyhdr}
\usepackage{graphicx}
\usepackage{colortbl}
\usepackage{titletoc,titlesec}
\usepackage{setspace}
\usepackage{indentfirst}
%\usepackage{natbib}
\usepackage[colorlinks=true, allcolors=black]{hyperref}
%\usepackage[brazilian,hyperpageref]{backref}
\usepackage[alf]{abntex2cite}
\usepackage{multirow} % https://www.ctan.org/pkg/multirow
\usepackage{float} % https://www.ctan.org/pkg/float
\usepackage{booktabs} % https://www.ctan.org/pkg/booktabs
\usepackage{enumitem} % https://www.ctan.org/pkg/enumitem
\usepackage{quoting} % https://www.ctan.org/pkg/quoting
\usepackage{epigraph}
\usepackage{subfigure}
\usepackage{anyfontsize}
\usepackage{caption}
\usepackage{adjustbox}
\usepackage{bm}
\usepackage{scalefnt}

%\usepackage[style=abnt]{biblatex}
%\bibliographystyle{plainnat}
\raggedbottom % https://latexref.xyz/_005craggedbottom.html

\newtheorem{teo}{Teorema}[section]
\newtheorem{lema}[teo]{Lema}
\newtheorem{cor}[teo]{Corolário}
\newtheorem{prop}[teo]{Proposição}
\newtheorem{defi}{Definição}
\newtheorem{exem}{Exemplo}

\newcommand{\titulo}{\large Título do relatório}
\newcommand{\autor}{Nome completo do autor (a)}
\newcommand{\orientador}{ }
%\newcommand{\coorientador}{ Prof(a). }
\newcommand{\mat}[1]{\mathbf{#1}}

\pagestyle{fancy}
\fancyhf{}
%\renewcommand{\headrulewidth}{0pt}
\setlength{\headheight}{16pt}
%C - Centro, L - Esquerda, R - Direita, O - impar, E - par
\fancyhead[RO, LE]{\thepage}
\renewcommand{\sectionmark}[1]{\markboth{#1}{}}

\titlecontents{section}[0cm]{}{\bf\thecontentslabel\ }{}{\titlerule*[.75pc]{.}\contentspage}
\titlecontents{subsection}[0.75cm]{}{\thecontentslabel\ }{}{\titlerule*[.75pc]{.}\contentspage}

\setcounter{secnumdepth}{3}
%\setcounter{tocdepth}{3}

\DeclareCaptionFormat{myformat}{ \centering \fontsize{10}{12}\selectfont#1#2#3}
\captionsetup{format=myformat}

\begin{document}


\begin{titlepage}
\begin{center}
\begin{figure}[h!]
	\centering
		\includegraphics[scale = 1.0]{imagens/unb.png}
	\label{fig:unb}
\end{figure}
{\bf \large Universidade de Brasília \\
\bf Departamento de Estatística}
\vspace{5cm}

\setcounter{page}{0}
\null
\textbf{\titulo}
\vspace{2.5cm}


\vspace{0.2cm}
\textbf{\autor}
\end{center}
\vspace{1.5cm}

\begin{flushright}
\begin{minipage}{7.5cm}
\parbox[t]{7.5cm}{Relatório apresentado para a disciplina nome-da-disciplina - EST0000 como parte dos requisitos necessários para aprovação.}
\end{minipage}
\end{flushright}

\vspace{5cm}

\begin{center}
{\bf{Brasília} \\ }
\bf{2024}
\end{center}

\end{titlepage}


%\include{rosto}

\pagenumbering{arabic}
\setcounter{page}{2}
\onehalfspacing




\setlength{\parindent}{1.5cm}
\setlength{\parskip}{0.2cm}
\setlength{\intextsep}{0.5cm}

\titlespacing*{\section}{0cm}{0cm}{0.5cm}
\titlespacing*{\subsection}{0cm}{0.5cm}{0.5cm}
\titlespacing*{\subsubsection}{0cm}{0.5cm}{0.5cm}
\titlespacing*{\paragraph}{0cm}{0.5cm}{0.5cm}

\titleformat{\paragraph}
{\normalfont\normalsize\bfseries}{\theparagraph}{1em}{}
%\titlespacing*{\paragraph}
%{0pt}{3.25ex plus ex minus .2ex}{1.5ex plus .2ex}



\pagenumbering{arabic}
\setcounter{page}{3}

\fancyhead[RE, LO]{\nouppercase{\emph\leftmark}}
%\fancyfoot[C]{Departamento de Estatística}

%Sumário
\tableofcontents

\newpage

\section{\textbf{Introdução}}\label{Sec1}
Aqui deve ser inserido a introdução para o relatório ...

A análise estatística apresentada neste relatório refere-se a...



\section{Título da seção}\label{Sec2}

Texto texto texto...

Por exemplo, esta seção pode ser utilizada para apresentar a análise descritiva dos dados.

\subsection{Título da subseção}

Texto texto texto... \\

aaaa...\\

{ \bf Exemplo de inserção de expressão matemática:}

O modelo de regressão linear simples (MRLS) é definido por
\begin{eqnarray*}
Y_i = \beta_0 + \beta_1x_i + \varepsilon_i, \ \ i=1, \ldots, n,
\end{eqnarray*}
em que  ...

%Ou ainda,
%$$Y_i = \beta_0 + \beta_1x_i + \varepsilon_i, \ \ i=1, \ldots, n,$$


A Tabela \ref{tab1} apresenta as estimativas, erros-padrão, estatísticas-$t$  e $p$-valores do teste  da nulidade dos coeficientes da regressão. A partir destes resultados, nota-se que ...
%Modelo de tabela:
\begin{table}[!htb]
	\centering
	\caption{Estimativas, erros padrão,  estatistica $t$ e $p$-valor sob os MRLS ajustados -- Dados nome.}\vspace{0.2cm}
	\label{tab1}
	\scalefont{0.80}
	\def\arraystretch{1} \begin{tabular}{rrrrrrrrrr}\hline
& \multicolumn{4}{c}{Modelo 1 }                                   && \multicolumn{4}{c}{Modelo 2 } \\  \cmidrule{2-5}\cmidrule{7-10}
& Estimativa   &Erro padrão &$t$   &$p$-valor                       && Estimativa  & Erro padrão &$t$   &$p$-valor     \\ \cmidrule{2-5}\cmidrule{7-10}
		{\it parâmetro}&&&&&&&&\\
		$\beta_0$        & $9.399$  &  $0.244$  & $38.423$  &  $<0.001$                       && $ 11.091$  &  $0.288$  & $38.482$ & $<0.001$      \\
		$\beta_1$        &  $0.255$  &  $0.010$  &  $25.239$   &  $<0.001$                       &&  $0.331$  &  $0.011$  &  $29.516$  &  $<0.001$ \\
  \cmidrule{2-5}\cmidrule{7-10}\\\hline
& \multicolumn{4}{c}{Modelo 3 }                                   && \multicolumn{4}{c}{Modelo 4 } \\  \cmidrule{2-5}\cmidrule{7-10}
& Estimativa   &Erro padrão &$t$   &$p$-valor                       && Estimativa  & Erro padrão &$t$   &$p$-valor     \\ \cmidrule{2-5}\cmidrule{7-10}
		{\it parâmetro}&&&&&&&&\\
		$\beta_0$        & $9.399$  &  $0.244$  & $38.423$  &  $<0.001$                       && $ 11.091$  &  $0.288$  & $38.482$ & $<0.001$      \\
		$\beta_1$        &  $0.255$  &  $0.010$  &  $25.239$   &  $<0.001$                       &&  $0.331$  &  $0.011$  &  $29.516$  &  $<0.001$ \\\hline
 \end{tabular}
\end{table}



\section{Título da seção}\label{Sec3}

Texto texto texto...

Forma de referenciar uma seção anterior:

Como discutido na Seção \ref{Sec3} ... \\

\subsection{Título da subseção}

Texto texto texto...

A Figura \ref{ENVELOPES1}  apresenta gráficos de probabilidade normal dos resíduos studentizados com envelopes simulados para cada modelo de regressão linear ajustado. A partir desta figura observa-se que...
%Modelo de figura acomodada em blocos:
\begin{figure}[!htb]
	\centering
	\subfigure[Modelo 1]{\includegraphics[width=6cm,height=5.5cm]{imagens/GPN1_ES.jpeg}}
	\qquad
	\subfigure[Modelo 2]{\includegraphics[width=6.0cm,height=5.5cm]{imagens/GPN2_ES.jpeg}}
    \qquad
	\subfigure[Modelo 3]{\includegraphics[width=6.0cm,height=5.5cm]{imagens/GPN1_ES.jpeg}}
    \qquad
	\subfigure[Modelo 4]{\includegraphics[width=6.0cm,height=5.5cm]{imagens/GPN1_ES.jpeg}}
	\caption{Gráficos de probabilidade normal dos resíduos studentizados com envelopes simulados sob os MRLS ajustados - Dados nome. }
	\label{ENVELOPES1}
\end{figure}

\section{Título da seção}\label{Sec4}

Texto texto...\\

Formas de referenciar materiais:

Segundo \citeonline{charnet} ...

Os modelos de regressão linear \cite{montgomery2012} são um conjunto de técnicas ...

Texto texto texto...

\subsection{Título da subseção}

Texto texto texto...

\subsection{Título da subseção}

Texto texto texto...

\section{Conclusões}

Aqui deve ser inserido a conclusão para o relatório ...\\


Neste relatório foi discutido...

\bibliography{ref.bib} %caso tenha referências a serem citadas


\end{document}
